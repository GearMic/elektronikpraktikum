%% packages
\documentclass{article}
\usepackage[a4paper, left=1.5cm, right=1.5cm, top=3.5cm]{geometry}
\usepackage[ngerman]{babel}
\usepackage{graphicx}
\usepackage{multicol}
\usepackage{amssymb}
\usepackage{titlesec}
\usepackage{wrapfig}
\usepackage{blindtext}
\usepackage{lipsum}
\usepackage{caption}
\usepackage{listings}
\usepackage{fancyhdr}
\usepackage{nopageno}
\usepackage{authblk}
\usepackage{amsmath} % tons of math stuff
\usepackage{mathtools} % e.g. alignment within matrix
%\usepackage{bm} % provides shorthand for bold in math mode
\usepackage{dsfont} % \mathds makes double stroke digits
\usepackage{esdiff} % provides \diff
%\usepackage[ISO]{diffcoeff}
\usepackage{xcolor}
\usepackage{csquotes} % e.g. provides \enquote
\usepackage{siunitx} % units
\usepackage{xcolor} % colored text
%\fancyhf[]{}


%% custom stuff
% own units
\DeclareSIUnit \VSS {\ensuremath{V_\mathrm{SS}}}
\DeclareSIUnit \VS {\ensuremath{V_\mathrm{S}}}
\DeclareSIUnit \Veff {\ensuremath{V_\mathrm{eff}}}
\DeclareSIUnit \Vpp {\ensuremath{V_\mathrm{pp}}}
\DeclareSIUnit \Vp {\ensuremath{V_\mathrm{p}}}
\DeclareSIUnit \VRMS {\ensuremath{V_\mathrm{RMS}}}
\DeclareSIUnit \ASS {\ensuremath{A_\mathrm{SS}}}
\DeclareSIUnit \AS {\ensuremath{A_\mathrm{S}}}
\DeclareSIUnit \Aeff {\ensuremath{A_\mathrm{eff}}}
\DeclareSIUnit \App {\ensuremath{A_\mathrm{pp}}}
\DeclareSIUnit \Ap {\ensuremath{A_\mathrm{p}}}
\DeclareSIUnit \ARMS {\ensuremath{A_\mathrm{RMS}}}

% change subsection numbering to capital letters
\newcommand{\subsectionAlph}{ \renewcommand{\thesubsection}{\arabic{section}.\Alph{subsection}} }
% change subsection numbering to lowercase letters
\newcommand{\subsectionalph}{ \renewcommand{\thesubsection}{\arabic{section}.\alph{subsection}} }
% own fig. that works with multicols
\newenvironment{Figure}
  {\par\medskip\noindent\minipage{\linewidth}}
  {\endminipage\par\medskip}
\newcommand*{\inputPath}{./plot} % prepend this command to the argument of all input commands
\graphicspath{ {./figure/} }


% own commands
% \newcommand{\rarr}{$\to\,$} %A$\,\to\,$B
\newcommand{\defc}{black}
\newcommand{\colorT}[2][blue]{\color{#1}{#2}\color{\defc}}
\newcommand{\redq}{\color{red}(?)\color{\defc}}
\newcommand{\question}[1]{\colorT[purple]{\textbf{(#1)}}}
\newcommand{\todo}[1]{\colorT[red]{\textbf{(#1)}}}
\newcommand{\mr}{\mathrm}

%% preparation
\begin{titlepage}
    \title{Elektronikpraktikum: Versuch 1 \\Ausbreitung von Signalen auf Leitungen}
    \author[1]{Marc Hauer\thanks{s??mhaue@uni-bonn.de}}
    \author[1]{Michael Vogt\thanks{s65mvogt@uni-bonn.de}}
    \affil[1]{Uni Bonn}
    %\date{\today}
\end{titlepage}


%% document
\begin{document}

\pagenumbering{gobble}
\maketitle
\tableofcontents
\newpage
\pagenumbering{arabic}

\pagestyle{fancy}
\fancyhead[R]{\thepage}
\fancyhead[L]{\leftmark}

\todo{Allgemeine Fragen beantworten können!}
Es sollen die Eigenschaften von Koaxialkabeln und die wellenartige Ausbreitung von Signalen durch diese Leiter untersucht werden.

\section{Theorie}
Ein Koaxkabel besteht aus einem inneren Leiter, der durch ein Dielektrikum von einem äußeren, zylinderförmigen Leiter getrennt ist.
Der äußere Leiter dient als Erdelektrode und/oder zur Abschirmung äußerer Felder. Durch diesen Aufbau hat das Kabel sowohl eine
Kapazität $C$, als auch eine Induktivität $I$, welche die Signalausbreitung maßgeblich beeinflussen. Diese lassen sich berechnen durch
\begin{equation}
  C = \epsilon_r \epsilon_0 l \frac{2\pi}{\ln\left(R_a/R_i\right)} \hspace*{1cm} L = \mu_r \mu_0 l \frac{\ln\left(R_a/R_i\right)}{2\pi}
\end{equation}
Hinzu kommen die charakteristischen Größen Verlustleitwert (\enquote{Ableitung}) $G$ und Widerstand $R$,
welche den Verlustwiderstand parallel zur Kapazität bzw. in Reihe zur Induktivität beschreiben.
All diese Größen wachsen linear mit der Länge des Kabels. Daher werden Längenunabhängige Größen eingeführt,
bei denen ein Hochkomma an das Formelzeichen gesetzt wird, z.B. $R' = R/l$ mit der Leiterlänge $l$.

Zur Veranschaulichung kann das Kabel als Reihenschaltung vieler kleiner LC-Glieder betrachtet werden (Abb. \ref{fig:lc-element}).
Durch die nichtverschwindende Ladezeit des Kondensators ist der Wert der Ausgangsspannung $U_\text{out}$ verzögert zur
Eingangsspannung $U_\text{in}$. Diese Verzögerung addiert sich auf und führt zu einer endlichen Ausbreitungsgeschwindigkeit des
Signals im Kabel.

\begin{figure}[h]
  \centering
  \includegraphics[width=0.3\textwidth]{LC_element}
  \caption{LC-Glied als elementarer Baustein des Koaxkabels}
  \label{fig:lc-element}
\end{figure}

Zur vollständigen mathematischen Beschreibung müssen außerdem Widerstand und Verlustleitwert beachtet werden.
Die Induktivität und Kapazität werden durch die Längsimpedanz $Z \coloneq i\omega L + R$
und Admittanz $Y \coloneq i\omega C + G$, bzw. dementsprechende infinitesimale Größen, ersetzt.
Daraus ergeben sich die DGLs
\begin{align}
  \diff[2]{U}{x} - \Upsilon^2U = 0, \enspace \diff{I}{x} = -U\cdot Y'\\
  \text{mit } \Upsilon^2 = Z'\cdot Y' \nonumber
\end{align}
Die Lösung, mit zusätzlicher Betrachtung der Zeitabhängigkeit, ist für die Spannung
\begin{equation}
  U(x, t) = U_\mr h (x, t) + U_\mr r (x, t) = (U_{\mr h0} e^{-\Upsilon x} + U_{\mr r0} e^{\Upsilon x}) e^{i\omega t} 
\end{equation}
\todo{wo kommt $\omega$ her?}
Es gibt also eine hinlaufende und eine rücklaufende, reflektierte Welle, und die Gesamtspannung entsteht aus Addition dieser Wellen.
Für den Strom ergibt sich eine ähnliche Gleichung, wobei jedoch die rücklaufende Welle einen im Verhältnis zu $U_{\mr r0}$ negativen Beitrag hat.

Das Verhältnis $\frac{U_\mr h}{I_\mr h}$ ist der \textit{Wellenwiderstand}
\begin{equation}
  Z = \frac{U_\mr h}{I_\mr h} = \sqrt\frac{R' + i\omega L'}{G' + i\omega C'}
\end{equation}
Dies ist also eine von der Kabellänge unabhängige Größe.
Die Amplitude der rücklaufenden Welle hängt ab vom Abschlusswiderstand
$R_\mr A$, mit dem das Kabelende ($x = l$) belastet wird. Es gilt für den \textit{Reflektionsfaktor} $r = U_{\mr r0} / U_{\mr h0}$
\begin{equation}
  r = \frac{R_\mr A - Z}{R_\mr A + Z}
\end{equation}
Für $R_A = Z$ \question{ist RA dann komplex?} ist $r = 0$. In einem solchen Fall wird die gesamte Energie der einlaufenden Welle wird über $R_\mr A$ umgesetzt
und es gibt keine reflektierte Welle. Dieser Zustand wird als \textit{Leitungsanpassung} bezeichnet. Im Leerlauf ($R_\mr A = \infty$)
ist $r=1$ und im Kurzschluss ($R_\mr A = 0$) $r=-1$.

Für die Phasen- bzw. Gruppengeschwindigkeit der einlaufenden Welle ergibt sich
\begin{equation}
  v_\mr{ph} = v_\mr{gr} = 1/\sqrt{L'C'} = c_0/\sqrt{\epsilon_r \mu_r}
\end{equation}
also gleicht die Geschwindigkeit der einer Wellenausbreitung im Dielektrikum zwischen den Leitern.



\section{Voraufgaben}
\begingroup
\subsectionAlph

\subsection{}
Die Verzögerungszeit im Kabel ist $t_\mr{delay} = \frac{l}{v_\mr{ph}} = l \sqrt{L'C'} = lc_0\sqrt{\epsilon_r\mu_r}$.
Sie hängt also vom Dielektrikum und von der Länge des Kabels ab. Zum Erhöhen von $t_\mr{delay}$ kann man das Kabel verlängern,
oder ein Dielektrikum mit größerem $\epsilon_r$ bzw. $\mu_r$ wählen.

\subsection{}
Da $Z=\sqrt\frac{L'}{C'}$ hat eine Verlängerung des Kabels keinen Einfluss auf den Wellenwiderstand.
Eine Erhöhung von $\mu_r$ bzw. $L'$ erhöht den Wellenwiderstand, eine Erhöhung von $\epsilon_r$ bzw. $C'$ verringert ihn.

\subsection{}
Die Leitung ist reflektionsfrei abgeschlossen und wirkt dadurch für die Signalquelle wie ein unendlich langes Kabel
mit Widerstand $Z$. Der Eingangswiderstand ist also unabhängig von der Länge. \question{Was ist ein Eingangswiderstand?}

\subsection{}
Ein Leiter mit $\frac{R_\mr a}{R_\mr i} = 2.3$ und $\epsilon_r = \mu_r = 1.5$ hat eine Phasengeschwindigkeit von
$v_\mr{ph} = \frac{c_0}{\sqrt{\epsilon_r \mu_r}} = \frac{c_0}{1.5} = \SI{2e8}{\m\per\s}$. Im verlustfreien Idealfall gilt
für den Wellenwiderstand $Z=\sqrt{\frac{L'}{C'}} = \sqrt{\frac{\mu_r \mu_0}{\epsilon_r \epsilon_0}} \frac{\ln(R_\mr a / R_\mr i)}{2\pi} = 120\pi \unit{\ohm} \frac{\ln{2.3}}{2\pi} \approx 49.97\unit{\ohm}$.
Die Verzögerungszeit pro Länge ist $\frac{t_\mr{delay}}{l} = 1/v_\mr{ph} = \SI{5}{\ns \per \m}$.

\endgroup


\section{Durchführung}
Zur Untersuchung des Verhaltens der Koaxialkabel sind kurze Spannungsimpulse nützlich. Diese erhalten wir durch differenzieren
eines Rechtecksignals mithilfe eines RC-Hochpassglieds. Zunächst wird das Verhalten dieser Impulse
mit der Schaltung in Abb. \ref{fig:aufgabe1} untersucht.
\question{Hat der CH1-Eingang zwei Kontakte? Wird der zweite Kontakt geerdet?}
\begin{figure}[h]
  \centering
  \includegraphics[width=0.4\textwidth]{aufgabe1}
  \caption{Aufbau für Versuchsaufgabe 1}
  \label{fig:aufgabe1}
\end{figure}

Das Oszi wird extern getriggert durch das dafür vorgesehene Signal des Signalgenerators. Wir beobachten die resultierende
Signalform bei einer $\SI{200}{\kilo\hertz}$-Rechteckschwingung. Anschließend wiederholen wir das gleiche mit einem $\SI{2.2}{\kilo\ohm}$-
Abschlusswiderstand am RC-Glied. \question{wo genau ist der Abschlusswiderstand?}

Wir wollen die Impulsform in einem (effektiv) an beiden Enden offenen Kabel untersuchen.
Dazu schließen wir an das Differenzierglied zwei $\SI{6}{\meter}$-Koaxkabel (auf Kabelrollen) hintereinander an.
Diese agieren effektiv wie $\SI{12}{\m}$ langes Kabel, bei dem in der Mitte das Signal abgegriffen werden kann (Abb. \ref{fig:aufgabe2}).
Am anderen Kanal des Oszillographen wird das Signal vor erreichen des Kabels abgegriffen. Damit die Eingangsseite des Kabels
\enquote{offen} ist, wird das Signal über einen hohen (Im Vergleich zum Wellenwiderstand des Kabels) Widerstand geleitet.
Wir verwenden ein $\SI{100}{\kilo\hertz}$-Rechtecksignal.
\begin{figure}[h]
  \centering
  \includegraphics[width=0.4\textwidth]{aufgabe2}
  \caption{Aufbau für Versuchsaufgabe 2}
  \label{fig:aufgabe2}
\end{figure}

Als nächstes Betrachten wir den Effekt verschiedener Abschlusswiderstände und Verzögerungszeiten mithilfe der Schaltung in Abb. \ref{fig:aufgabe3}.
Wir betrachten das Signal mit / ohne Abschlusswiderstand ($\SI{50}{\ohm}$) und mit offenem bzw. kurzgeschlossenem Verzögerungskabel (oben rechts).
Wir untersuchen die Unterschiede bei leichter Variation der Frequenz.
\begin{figure}[h]
  \centering
  \includegraphics[width=0.4\textwidth]{aufgabe3}
  \caption{Aufbau für Versuchsaufgabe 3}
  \label{fig:aufgabe3}
\end{figure}

Zuletzt entfernen wir das Differenzierglied sowie den Anpasswiderstand \question{welcher ist das?}, um den Effekt eines Kabels auf
das Rechtecksignal zu betrachten. Zur Anpassung des Eingangswiderstands \todo{falsches Wort?} wird ein $\SI{2450}{\ohm}$-Widerstand vorgeschaltet.
Die Verzögerungszeit durch ein Kabel kann ausgenutzt werden, um die Länge des Rechteckpulses zu begrenzen.
Wir betrachten das Oszillogramm mit offenem bzw. kurzgeschlossenem Verzögerungskabel \todo{stimmt das so? Ist das Verzögerungskabel das Klippkabel?}.
Wir variieren erneut die Frequenz und betrachten die Dämpfung des Signals. Wir ermitteln die spezifische Dämpfung, also
den Betrag der Dämpfung (in \unit{\decibel}) pro Kabellänge, für das verwendete Klippkabel.
\begin{figure}[h]
  \centering
  \includegraphics[width=0.4\textwidth]{aufgabe4}
  \caption{Aufbau für Versuchsaufgabe 4}
  \label{fig:aufgabe4}
\end{figure}
\end{document}
